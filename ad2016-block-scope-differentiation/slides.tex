% \documentclass{beamer}
\documentclass[xcolor=dvipsnames]{beamer}
%% \usecolortheme[named=Blue]{structure}
\setbeamersize{text margin left=30mm, text margin right=30mm}
\useoutertheme{infolines}
\usetheme[height=7mm]{Rochester}
\setbeamertemplate{items}[ball]
\setbeamertemplate{blocks}[rounded][shadow=true]
\setbeamertemplate{navigation symbols}{}

\usepackage[utf8x]{inputenc}
\usepackage{default}
\usepackage[english]{babel}
\usepackage{geometry}
%% \usepackage{fullpage}
\usepackage{amsmath, amsthm, amssymb}
\usepackage{listings}
\usepackage{pxfonts}
%% \usepackage{color}
%% \usepackage{graphicx}
%% \usepackage{natbib}
%% \usepackage{array}
%% \usepackage{booktabs}
%% \usepackage{tabu}
%% \usepackage[utf8]{inputenc}
%% \usepackage{fancyhdr}
%% \usepackage{float}
%% \usepackage{subfigure}
%% \usepackage{titlesec}


\definecolor{lstgray}{gray}{0.97}
\lstset{ %
  escapechar=@,
  language=C++,
  basicstyle=\footnotesize\ttfamily,
  %% basicstyle=\ttfamily,
  %% keywordstyle=\color{blue}\ttfamily,
  keywordstyle=\bfseries,
  stringstyle=\color{red}\ttfamily,
  commentstyle=\color{OliveGreen}\ttfamily,
  morecomment=[l][\color{magenta}]{\#},
  backgroundcolor=\color{lstgray},
  %% keywordstyle=\color{red},
  frame=f,
  frameround=ffff,
  tabsize=2,
  breaklines=true,
  breakatwhitespace=false,
  showspaces=false,
  showstringspaces=false,
  xleftmargin=5pt,
  xrightmargin=5pt,
  morekeywords={in,out,ref,auto,inout,import,ushort,scope,exit,mixin}
}

\def\redcolor{\color{red}}
\def\bluecolor{\color{blue}}
\def\blackcolor{\color{black}}


\def\sectionname{\translate{Section}}
\def\insertsectionnumber{\arabic{section}}
\setbeamertemplate{section page}
{
  \begin{centering}
    \begin{beamercolorbox}[sep=4pt,center]{part title}
      \usebeamerfont{section title}\insertsection\par
    \end{beamercolorbox}
  \end{centering}
}
\def\sectionpage{\usebeamertemplate*{section page}}


\AtBeginSection{\frame{\sectionpage}}


\title{Block Scope Differentiation}
\subtitle{AD2016 Oxford}
%% \author{\texorpdfstring{Author\newline\url{email@email.com}}{Dominic Jones}}
\author{Dominic Jones}
%% \author{Author\\{\tiny email@email.com}}
\institute{\texttt{dominic.jones@gmx.co.uk}}
\date{London, September 2016}


\begin{document}
\begin{frame}[plain]
  \titlepage
\end{frame}


\section{Aim}


\begin{frame}{An additional helping hand}
\begin{itemize}
\item The aim is to have the fastest possible differentiation tool \vspace{5mm}
\item In \texttt{C++}, with 2011 standard available \vspace{5mm}
\item Which can integrate with existing framework \vspace{5mm}
\item Does not introduce unusual syntax \vspace{5mm}
\item A small tool which is easy to use selectively \vspace{5mm}
\end{itemize}
\end{frame}

\begin{frame}[fragile]{Target}
\begin{itemize}
\item Modern, commercial, CFD software \vspace{2mm}
\item Adjoint differentation of coupled Navier-Stokes solver \vspace{2mm}
\item Calculation of element cell terms via face-cell loops \vspace{2mm}
\end{itemize}
\begin{lstlisting}
for (f = 0; f != N; ++f)
{
  // cache input fields
  const Vector<3,double> A_f{A[f]};
  const Vector<3,float> U_f{U[f]};
  const double rho_f{rho[f]};

  // compute local values
  const double spd{dot(U_f, A_f)};
  const float flux{rho_f * spd};

  // write results
  R[f](0) += flux * U_f;
  R[f](1) -= flux * U_f;
}
\end{lstlisting}
\end{frame}


\section{Method}


\begin{frame}[fragile]{\emph{l-value} destructor}
\begin{itemize}
\item If the loop block does not have nested scopes then the reverse sequence of destructor operations could be harnessed \vspace{2mm}
\end{itemize}
\begin{lstlisting}
@\aftergroup\bluecolor@mode@\aftergroup\blackcolor@ = ADJOINT;
{
  ...

  // compute local values
  const @\aftergroup\bluecolor@Drv@\aftergroup\blackcolor@<@\aftergroup\bluecolor@mode@\aftergroup\blackcolor@,double> spd{dot(U_f, A_f)};
  const @\aftergroup\bluecolor@Drv@\aftergroup\blackcolor@<@\aftergroup\bluecolor@mode@\aftergroup\blackcolor@,float> flux{rho_f * spd};

  ...
  @\aftergroup\redcolor@
  ~flux() { rho_f.adj() += spd.pri() * flux.adj();
            spd.adj() += rho_f.pri() * flux.adj(); }

  ~spd() { U_f.adj() += A_f.pri() * spd.adj();
           A_f.adj() += U_f.pri() * spd.adj(); }@\aftergroup\blackcolor@
}
\end{lstlisting}
\end{frame}


\begin{frame}[fragile]{Caching the expression}
\begin{itemize}
\item With a reasonable guess of the {\color{blue}expression size}, a copy could be made during construction and then evaluated later by some {\color{red}engine} \vspace{2mm}
\end{itemize}
\begin{lstlisting}
{
  ...

  const Drv<mode,float,@\aftergroup\bluecolor@expr_size@\aftergroup\blackcolor@> flux{@\aftergroup\bluecolor@rho_f * spd@\aftergroup\blackcolor@};

  ...

  @\aftergroup\redcolor@~flux()
  {
    (*this->_adjointExpression)(this->_expr, this->_adj);
  }@\aftergroup\blackcolor@
}
\end{lstlisting}
\end{frame}


\begin{frame}[fragile]{Typelessness}
\begin{itemize}
\item At construction, the {\color{red}expression} gets copied to a local array and a {\color{blue}function pointer} to the adjoint expression engine is stored \vspace{2mm}
\end{itemize}
\begin{lstlisting}
template<Mode m, typename T, int s>
class Drv<m,ExprCache<T,s> > : public Drv<m,T>
{
  using AdjointExpression_t =
    void(*)(void const * const, T const &);

  // constructor
  template<typename Expr_t> Drv(Expr_t const &expr)
    : Drv<m,T>(primalExpression<Expr_t>(expr))
    , @\aftergroup\redcolor@_expr(memcpy(sizeof(expr), expr))@\aftergroup\blackcolor@
    , @\aftergroup\bluecolor@_adjointExpression(&adjointExpression<Expr_t,T>)@\aftergroup\blackcolor@
  {}

  // members
  @\aftergroup\redcolor@std::array<char,s> _expr;@\aftergroup\blackcolor@
  @\aftergroup\bluecolor@AdjointExpression_t _adjointExpression;@\aftergroup\blackcolor@
}
\end{lstlisting}
\end{frame}


\begin{frame}[fragile]{Type retention via \texttt{auto}}
\begin{itemize}
\item To facilitate \texttt{auto}, an {\color{blue}\texttt{assign}} function is required to build the correct \emph{l-value} type \vspace{2mm}
\end{itemize}
\begin{lstlisting}
{
  ...

  // compute local values
  const auto flux{@\aftergroup\bluecolor@assign(@\aftergroup\blackcolor@rho_f * spd@\aftergroup\bluecolor@)@\aftergroup\blackcolor@};

  ...

  @\aftergroup\redcolor@~flux()
  {
    adjointExpression<Expr_t,T>(this->_expr, this->_adj);
  }@\aftergroup\blackcolor@
}
\end{lstlisting}
\end{frame}


\begin{frame}{Expression context}
\begin{itemize}
\item Separate the context of how an expression is to be evaluated from the expression itself \vspace{5mm}
\item Made possible with user-defined types and operator overloading \vspace{5mm}
\item But needs to support arithmetic with mixed types ({\color{blue}\texttt{float}, \texttt{double}, \texttt{Vector<N,T>}, \texttt{Tensor<N,T>}}) \vspace{5mm}
\item Must co-exist with other operator overloading tools ({\color{blue}\texttt{PETE}}, Los Alamos National Laboratory) \vspace{5mm}
\end{itemize}
\end{frame}


\begin{frame}[fragile]{Expression tree}
\begin{itemize}
\item Nodes templated on operator, result and argument types \vspace{2mm}
\item Sub-nodes may be passive (non-differentiable) or active \vspace{2mm}
\item Sub-nodes may be owned by value or by reference \vspace{2mm}
\end{itemize}
\begin{lstlisting}
// expression
@\aftergroup\redcolor@dot(U_f, A_f);@\aftergroup\blackcolor@

// expression type
@\aftergroup\bluecolor@Binary<Dot,
       double,
       Drv<mode,Vector<3,float>>,
       Drv<mode,Vector<3,double>>
      >@\aftergroup\blackcolor@
\end{lstlisting}
\end{frame}


\begin{frame}[fragile]{Writing to the inputs}
\begin{itemize}
\item A differentiatable type retains the address either to its own adjoint value else to the adjoint value it was constructed with \vspace{2mm}
\end{itemize}
\begin{lstlisting}
template<Mode m, typename T>
class Drv : public ExpressionNode<m,T,Drv<T> >
{
  Drv(T const &pri, @\aftergroup\bluecolor@T &drv@\aftergroup\blackcolor@)
    : _pri(pri), _drv(drv), @\aftergroup\redcolor@_adj(drv)@\aftergroup\blackcolor@
  {}

  T const &pri() const { return _pri; }
  void adj(T const &rhs) const { @\aftergroup\redcolor@_adj += rhs;@\aftergroup\blackcolor@ }

  T _pri, _drv;
  @\aftergroup\redcolor@T &_adj;@\aftergroup\blackcolor@
};
\end{lstlisting}
\end{frame}


\begin{frame}[fragile]{Function signature}
\begin{itemize}
\item {\color{red}Mutable} and {\color{blue}immutable} types have slightly different syntax \vspace{2mm}
\end{itemize}
\begin{lstlisting}
template<Mode mode>
void
evaluate(@\aftergroup\bluecolor@const@\aftergroup\blackcolor@ Drv<mode,double> & A,
         @\aftergroup\bluecolor@const@\aftergroup\blackcolor@ Drv<mode,double> & B,
         Drv<mode,double@\aftergroup\redcolor@&@\aftergroup\blackcolor@> Z)
{
  const Drv<mode,double,10> T0{A * B};
  const Drv<mode,double,15> T1{reciprocal(A + B)};
  Z = T0 * T1;
}
\end{lstlisting}
\end{frame}


\begin{frame}[fragile]{Function invocation}
\begin{itemize}
\item Adjoint evaluation is no more than the function call \vspace{2mm}
\end{itemize}
\begin{lstlisting}
double a_pri{3}, b_pri{4};

double a_adj{0}, b_adj{0};
double @\aftergroup\bluecolor@z_adj{1};@\aftergroup\blackcolor@

evaluate(Drv<mode,double>{a_pri, @\aftergroup\redcolor@a_adj@\aftergroup\blackcolor@},
         Drv<mode,double>{b_pri, @\aftergroup\redcolor@b_adj@\aftergroup\blackcolor@},
         Drv<mode,double&>{@\aftergroup\bluecolor@z_adj@\aftergroup\blackcolor@});

std::cout << @\aftergroup\redcolor@a_adj@\aftergroup\blackcolor@ << std::end;
std::cout << @\aftergroup\redcolor@b_adj@\aftergroup\blackcolor@ << std::end;
\end{lstlisting}
\end{frame}


\section{Testing}


\begin{frame}[fragile]{Harmonic function}
\begin{itemize}
\item Test case used by NAG \vspace{5mm}
\item 5 inputs, 1 output, 100 lines \vspace{5mm}
\item \verb|github.com/DominicJones/AD2016_Oxford| \vspace{5mm}
\item Five approaches to evaluating the adjoint: \vspace{2mm}
  \begin{enumerate}
  \item Adept AD operator overloading tool (13.3x) \vspace{2mm}
  \item Tapenade AD source transformation tool (1.9x) \vspace{2mm}
  \item {\color{blue}typeless expression caching (5.8x)} \vspace{2mm}
  \item {\color{blue}typed expression caching (using \texttt{auto}) (5.8x)} \vspace{2mm}
  \item {\color{red}naive use of \texttt{auto} (320x)} \vspace{2mm}
\end{enumerate}
\item timings are median average of 50 evaluations, 100,000 iterations per evaluation (g++ 5.1 -O3, Intel Xeon E5-2650)
\end{itemize}
\end{frame}


%% mixed types
%% different approaches
%% branch re-evaluation


\section{Further work}


\begin{frame}[fragile]{Removing the expression copying}
\begin{itemize}
\item Copying every expression so that it can be used in the destructor is the principle hit on performance \vspace{5mm}
\item Instead, make the expression nodes perform the adjoint evaluation in \emph{their} destructors \vspace{5mm}
\item \texttt{auto} must be used in place of an \emph{l-value} type \vspace{5mm}
\end{itemize}
\end{frame}


\begin{frame}[fragile]{Supporting \texttt{if} blocks}
\begin{itemize}
\item Already possible, but tree must always own sub-nodes by value \vspace{5mm}
\item Within the nested scope, naive use of \texttt{auto} is necessary \vspace{5mm}
\end{itemize}
\end{frame}


\begin{frame}[fragile]{\texttt{auto} return type}
\begin{itemize}
\item With the 2014 standard, expression types can be returned from functions \vspace{5mm}
\item Removes the need to maintain function call back pointers \vspace{5mm}
\item But the expression must hold copies of local variables, rather than references \vspace{5mm}
\end{itemize}
\end{frame}


\end{document}
