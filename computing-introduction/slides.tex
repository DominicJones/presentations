% \documentclass{beamer}
\documentclass[xcolor=dvipsnames]{beamer}
%% \usecolortheme[named=Blue]{structure}
\setbeamersize{text margin left=30mm, text margin right=30mm}
\useoutertheme{infolines}
\usetheme[height=7mm]{Rochester}
\setbeamertemplate{items}[ball]
\setbeamertemplate{blocks}[rounded][shadow=true]
\setbeamertemplate{navigation symbols}{}

\usepackage[utf8x]{inputenc}
\usepackage{default}
\usepackage[english]{babel}
\usepackage{geometry}
%% \usepackage{fullpage}
\usepackage{amsmath, amsthm, amssymb}
\usepackage{listings}
%% \usepackage{color}
%% \usepackage{graphicx}
%% \usepackage{natbib}
%% \usepackage{array}
%% \usepackage{booktabs}
%% \usepackage{tabu}
%% \usepackage[utf8]{inputenc}
%% \usepackage{fancyhdr}
%% \usepackage{float}
%% \usepackage{subfigure}
%% \usepackage{titlesec}


\definecolor{lstgray}{gray}{0.9}
\lstset{ %
  language=C++,
  basicstyle=\footnotesize\ttfamily,
  %% basicstyle=\ttfamily,
  keywordstyle=\color{blue}\ttfamily,
  stringstyle=\color{red}\ttfamily,
  commentstyle=\color{green}\ttfamily,
  morecomment=[l][\color{magenta}]{\#},
  backgroundcolor=\color{lstgray},
  %% keywordstyle=\color{red},
  frame=single,
  frameround=tttt,
  tabsize=2,
  breaklines=true,
  breakatwhitespace=false,
  showspaces=false,
  showstringspaces=false,
  xleftmargin=20pt,
  xrightmargin=20pt,
  morekeywords={in,out,inout,import,ushort}
}

%% \AtBeginSection{\frame{\sectionpage}}


\title{Computing}
\subtitle{A difficult introduction}
\author{Dr D P Jones}
\institute{The Cedars School}
\date{October 2013}


\begin{document}
\begin{frame}[plain]
  \titlepage

  \begin{center}
    \texttt{https://github.com/{\color{red}TheCedarsSchool}/{\color{blue}computing}}
  \end{center}
\end{frame}


\begin{frame}{Language of the machine}
  %% \begin{quote}
  %%   ``It is essential to have a beard if one is to be a great computer scientist'' -anon
  %% \end{quote}
  \begin{figure}
    \centering
    \includegraphics[width=0.4\textwidth]{../figures/dennisRitchie.jpg}
  \end{figure}
  \begin{center}
    Dennis Ritchie (1941 - 2011); creator of C
  \end{center}
\end{frame}


\begin{frame}{Overview}
  \tableofcontents
\end{frame}


\section{The problem with computers}


\begin{frame}{Counting}
  \begin{definition}
    A natural number is any integer (whole number) greater than or equal to zero
  \end{definition}\pause

  \begin{equation*}
    c \ge 0
  \end{equation*}\pause

  \begin{example}
    \begin{itemize}
    \item 0
    \item 145
    \item 68370912
    \end{itemize}
  \end{example}
\end{frame}


\begin{frame}{But knowing when to stop}
  \begin{definition}
    A finite natural number is any natural number less than or equal to some maximum value
  \end{definition}\pause

  \begin{equation*}
    0 \le c \le max
  \end{equation*}\pause

  \begin{example}
    \begin{itemize}
    \item 00000000
    \item 00000145
    \item 68370912
    \end{itemize}
    where 99999999 is the maximum
  \end{example}
\end{frame}


\begin{frame}[fragile]
  \frametitle{Computers can add too}
  \begin{lstlisting}
    #include "stdio.h"

    int main()
    {
      unsigned short a =  5000;
      unsigned short b = 61000;
      unsigned short c;

      c = a + b;

      printf("%d", c);
    }
  \end{lstlisting}
\end{frame}


\begin{frame}[fragile]
  \frametitle{When it all goes wrong}
  Ariane 5 rocket, Flight 501, 4 June 1996 exploded
  \begin{figure}
    \centering
    \includegraphics[width=0.6\textwidth]{../figures/ariane_v501.jpg}
  \end{figure}
  because a line of its code tried to store a number which was too big.
\end{frame}


\begin{frame}[fragile]
  \frametitle{So maybe ``max'' was important after all...}
  One way find ``max'' is to keep adding 1 until something strange happens
  \vspace{5mm}
  \begin{lstlisting}
    #include "stdio.h"

    int main()
    {
      unsigned short c = 0;
      unsigned short c_plus_1 = 1;

      while (c_plus_1 == c + 1)
      {
        c = c + 1;
        c_plus_1 = c_plus_1 + 1;
      }

      printf("%d", c);
    }
  \end{lstlisting}
\end{frame}


\begin{frame}[fragile]
  \frametitle{A 12 hour clock for a 24 hour day}
  What time is 14:00 on a clock?
  \begin{figure}
    \centering
    \includegraphics[width=0.25\textwidth]{../figures/clock.png}
  \end{figure}
  \begin{lstlisting}
    print(14 % 12)
  \end{lstlisting}
\end{frame}


\begin{frame}[fragile]
  \frametitle{Back to the mysterious addition}
  \begin{lstlisting}
    a =  5000
    b = 61000

    c = a + b

    print(c % 65535)
  \end{lstlisting}
\end{frame}


\section{Counting up to one}


\begin{frame}{So can we not have bigger numbers?}
  1988: Sega Genesis (16-bit console)
  \begin{center}
    \includegraphics[width=0.65\textwidth]{../figures/segaGenesis.jpg}
  \end{center}
\end{frame}


\begin{frame}{So can we not have bigger numbers?}
  1994: Sony PlayStation (32-bit console)
  \begin{center}
    \includegraphics[width=0.65\textwidth]{../figures/playStation.jpg}
  \end{center}
\end{frame}


\begin{frame}{So can we not have bigger numbers?}
  1996: Nintendo 64 (64-bit console)
  \begin{center}
    \includegraphics[width=0.65\textwidth]{../figures/nintendo64.jpg}
  \end{center}
\end{frame}


\begin{frame}{Everything is made of bits}
  \begin{itemize}
  \item binary digits (bits) are a like ordinary digits ({\color{red}0} to {\color{red}9}), but there are less of them.\newline In fact there are only {\color{blue}two}: {\color{red}0} and {\color{red}1}.\pause \vspace{5mm}
  \item {\color{red}105} is ``{\color{red}1} times {\color{blue}100}, plus {\color{red}0} times {\color{blue}10}, plus {\color{red}5} times {\color{blue}1}''.\pause \vspace{5mm}
  \item More importantly, it is also 01101001, expressed in (8-bit) binary.\pause \vspace{5mm}
  \item And the maximum an 8-bit number can represent is 11111111,\newline which is 255.
  \end{itemize}
\end{frame}


\begin{frame}[fragile]
  \frametitle{How many bits does my number need?}
  \begin{lstlisting}
    #include "stdio.h"

    int main()
    {
      unsigned long c = 0;
      unsigned long c_max = 65535;

      unsigned long order  = 1;
      unsigned long n_bits = 0;

      while (c < c_max)
      {
        c = c + order;
        order = 2 * order;
        n_bits = n_bits + 1;
      }

      printf("%d", n_bits);
    }
  \end{lstlisting}
\end{frame}


\section{Roman secret messaging service}


\begin{frame}{Old-style encryption}
  \begin{figure}
    \centering
    \includegraphics[width=0.3\textwidth]{../figures/juliusCaesar.jpg}
  \end{figure}
  \begin{center}
    Julius Caesar (100 - 44BC)
  \end{center}
\end{frame}


\begin{frame}{How it works (encryption)}
  \begin{figure}
    \centering
    \includegraphics[width=0.9\textwidth]{../figures/cypherSchematic.pdf}
  \end{figure}
\end{frame}


\begin{frame}{How it works (decryption)}
  \begin{figure}
    \centering
    \includegraphics[width=0.9\textwidth]{../figures/uncypherSchematic.pdf}
  \end{figure}
\end{frame}


\begin{frame}{Julius didn't trust everyone...}
  \begin{figure}
    \centering
    \includegraphics[width=0.7\textwidth]{../figures/caesarCypher.jpg}
  \end{figure}
  \begin{center}
    Key = 25: A becomes Z, B becomes A,...
  \end{center}
\end{frame}


\begin{frame}[fragile]
  \frametitle{...only his closest pals}
  \begin{lstlisting}
    def cypher(message, key = 13):
      result = ""
      a_value = ord('a')

      for letter in message:

        value = ord(letter)
        value_0 = value - a_value

        if letter.islower():
          value_0 = (value_0 + key) % 26

        value = value_0 + a_value
        result = result + chr(value)

      return result

    print(cypher("nyrn vnpgn rfg"))
  \end{lstlisting}
\end{frame}


\begin{frame}{ASCII Tables}
  \begin{figure}
    \centering
    \includegraphics[width=0.3\textwidth]{../figures/ascii97.png}
  \end{figure}
\end{frame}
\end{document}
