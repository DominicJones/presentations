% \documentclass{beamer}
\documentclass[xcolor=dvipsnames]{beamer}
%% \usefonttheme[onlymath]{serif}
\usefonttheme{professionalfonts}
%% \usecolortheme[named=Blue]{structure}
\setbeamersize{text margin left=30mm, text margin right=30mm}
\useoutertheme{infolines}
%% \usetheme[height=7mm]{Rochester}
\usetheme{Pittsburgh}
\setbeamertemplate{items}[ball]
\setbeamertemplate{blocks}[rounded][shadow=true]
\setbeamertemplate{navigation symbols}{}

\usepackage[utf8x]{inputenc}
\usepackage{default}
\usepackage[english]{babel}
\usepackage{geometry}
%% \usepackage{fullpage}
\usepackage{amsmath, amsthm, amssymb}
\usepackage{listings}
\usepackage{pxfonts}
%% \usepackage{color}
%% \usepackage{graphicx}
%% \usepackage{natbib}
%% \usepackage{array}
%% \usepackage{booktabs}
%% \usepackage{tabu}
%% \usepackage[utf8]{inputenc}
%% \usepackage{fancyhdr}
%% \usepackage{float}
%% \usepackage{subfigure}
%% \usepackage{titlesec}

\setbeamertemplate{headline}{}
\setbeamertemplate{footline}[frame number]{}
\setbeamertemplate{navigation symbols}{}
\setbeamertemplate{footline}{}

\renewcommand{\chaptername}{}
\renewcommand{\bibname}{References}
\newcommand{\pluseq}{\:+\!\!=}
\newcommand{\minuseq}{\:-\!\!=}
\newcommand{\mrm}[1]{\mathrm{#1}}
\newcommand{\bsym}[1]{\boldsymbol{#1}}
\newcommand{\abs}[1]{\lvert#1\rvert}
\newcommand{\norm}[1]{\lVert#1\rVert}

\newcommand{\totalderiv}[2]{\frac{\mrm{d} #1}{\mrm{d} #2}}
\newcommand{\partialderiv}[2]{\frac{\partial #1}{\partial #2}}

\newcommand{\totalderivT}[2]{\totalderiv{#1}{#2}^{\mrm{T}}}
\newcommand{\partialderivT}[2]{\partialderiv{#1}{#2}^{\mrm{T}}}


\setcounter{MaxMatrixCols}{20}

\def\CCT{{C\nolinebreak[4]\hspace{-.05em}\raisebox{.4ex}{\tiny\bf ++}}}
\def\CC{{C\nolinebreak[4]\hspace{-.05em}\raisebox{.4ex}{\small\bf ++}}}


\definecolor{lstgray}{gray}{0.93}
\lstset{ %
  escapechar=@,
  language=C++,
  basicstyle=\footnotesize\ttfamily,
  %% basicstyle=\ttfamily,
  %% keywordstyle=\color{blue}\ttfamily,
  keywordstyle=\bfseries,
  stringstyle=\color{red}\ttfamily,
  commentstyle=\color{OliveGreen}\ttfamily,
  morecomment=[l][\color{red}]{\#},
  backgroundcolor=\color{lstgray},
  %% keywordstyle=\color{red},
  frame=f,
  frameround=ffff,
  tabsize=2,
  breaklines=true,
  breakatwhitespace=false,
  showspaces=false,
  showstringspaces=false,
  xleftmargin=5pt,
  xrightmargin=5pt,
  morekeywords={in,out,ref,auto,inout,import,ushort,scope,exit,mixin,decltype,varid,sizeof}
}

\def\redcolor{\color{red}}
\def\bluecolor{\color{blue}}
\def\blackcolor{\color{black}}
\def\graycolor{\color{gray}}
\def\greencolor{\color{OliveGreen}}


\def\sectionname{\translate{Section}}
\def\insertsectionnumber{\arabic{section}}
\setbeamertemplate{section page}
{
  \begin{centering}
    \begin{beamercolorbox}[sep=4pt,center]{part title}
      \usebeamerfont{section title}\insertsection\par
    \end{beamercolorbox}
  \end{centering}
}
\def\sectionpage{\usebeamertemplate*{section page}}


\AtBeginSection{\frame{\sectionpage}}

\usepackage{caption}
\captionsetup[figure]{labelformat=empty}


\title{Aristotelian Proof}
\subtitle{The Unmoved Mover}
%% \author{\texorpdfstring{Author\newline\url{email@email.com}}{Dominic Jones}}
\author{Dominic Jones}
%% \author{Author\\{\tiny email@email.com}}
%% \institute{\texttt{dominic.jones@gmx.co.uk}}
%% \date{London, January 2018}


\begin{document}
\begin{frame}[plain]
  \titlepage
\end{frame}

\begin{frame}{}
\begin{figure}
  \centering
  \includegraphics[width=0.99\textwidth]{unmoved_mover}
\end{figure}
\end{frame}

\section{At least some things change}

\begin{frame}{From becoming cold or fat\ldots}
\begin{figure}
  \centering
  \begin{columns}
    \column{0.5\textwidth}
    \centering
    \caption {Qualitative change}
    \includegraphics[width=0.99\textwidth]{tea}
    \column{0.5\textwidth}
    \centering
    \caption {Quantitative change}
    \includegraphics[width=0.99\textwidth]{cat}
  \end{columns}
\end{figure}
\end{frame}

\begin{frame}{or to dropping and dying\ldots}
\begin{figure}
  \centering
  \begin{columns}
    \column{0.5\textwidth}
    \centering
    \caption {Change of location}
    \includegraphics[width=0.99\textwidth]{leaf}
    \column{0.5\textwidth}
    \centering
    \caption {Substantial change}
    \includegraphics[width=0.99\textwidth]{fox}
  \end{columns}
\end{figure}
\end{frame}

\begin{frame}{or going from ignorance to knowledge}
\begin{figure}
  \centering
  \begin{columns}
    \column{0.5\textwidth}
    \centering
    \caption {Change of mind}
    \includegraphics[width=0.99\textwidth]{pills}
    \column{0.5\textwidth}
    \centering
    \caption {Casting doubt}
    \includegraphics[width=0.99\textwidth]{stairs}
  \end{columns}
\end{figure}
\end{frame}


\begin{frame}[fragile]{Change}
  \begin{enumerate}
  \item There is at least some change in something \vspace{5mm}
  \item To undergo some change, the actual thing being changed must in some way be in potential the thing it becomes \vspace{5mm}
  \item Things may change are at once actually something but also potentially something else \vspace{5mm}
  \item A thing that changes cannot change exclusively by itself \vspace{5mm}
  \end{enumerate}
\end{frame}


\section{A change needs a changer}


\begin{frame}{}
\begin{figure}
  \centering
  \begin{columns}
    \column{0.5\textwidth}
    \centering
    \caption {This can be many things\ldots}
    \includegraphics[width=0.99\textwidth]{marble}
    \column{0.5\textwidth}
    \centering
    \caption {through an agent of change}
    \includegraphics[width=0.99\textwidth]{david}
  \end{columns}
\end{figure}
\end{frame}

\begin{frame}[fragile]{Agent}
  \begin{enumerate}
  \item Potential `David' cannot make itself, therefore an agent of change is required \vspace{5mm}
  \item That agent must be actual to bring about the change, therefore what is actual is \emph{prior} to what is potential \vspace{5mm}
    \item This kind of \emph{linear} change could go on forever\ldots \vspace{5mm}
  \end{enumerate}
\end{frame}

\begin{frame}{\ldots or could go back forever}
\begin{figure}
  \centering
  \begin{columns}
    %% \column{0.5\textwidth}
    %% \centering
    %% \caption {Can be many things...}
    %% \includegraphics[width=0.99\textwidth]{marble}
    \column{0.99\textwidth}
    \centering
    %% \caption {(but this is irrelevant)}
    \includegraphics[width=0.99\textwidth]{big_bang}
  \end{columns}
\end{figure}
  \begin{enumerate}
  \item Changes in time could \emph{metaphysically} have no starting point (no first member) \vspace{5mm}
  \item A linear series of changes is not \emph{at once} hierarchically dependent \vspace{5mm}
  \end{enumerate}
\end{frame}


\section{`That's just way it is' - this is also change}


\begin{frame}[fragile]{Hierarchy of the actual}
  \begin{enumerate}
  \item Things that change are described as having potency and act \vspace{5mm}
  \item But something that can change, and is the way it is \emph{here and now}, is also the actualisation of a potential \vspace{5mm}
    \item This kind of change is hierarchical: if there is no inherent power to be the way it is, this power must be derived
  \end{enumerate}
\end{frame}

\begin{frame}{An instrument of real power}
\begin{figure}
  \centering
  \begin{columns}
    \column{0.5\textwidth}
    \centering
    \caption {Not possible!}
    \includegraphics[width=0.99\textwidth]{levitation2}
    \column{0.5\textwidth}
    \centering
    \caption {Everyday occurrence}
    \includegraphics[width=0.99\textwidth]{levitation}
  \end{columns}
\end{figure}
\end{frame}


\begin{frame}[fragile]{First in line}
  \begin{enumerate}
  \item She cannot hold herself in the air by herself \vspace{5mm}
  \item The power to do so is imparted through the great `Kellar', despite being \emph{merely} an instrument \vspace{5mm}
  \item She is \emph{potentially} flat on the ground but for the \emph{act} of the intermediary \vspace{5mm}
  \item But not `and so on forever': something \emph{actual} must terminate this imparted power \vspace{5mm}
  \end{enumerate}
\end{frame}


\section{Something that changes must first exist}


\begin{frame}{(But not everything can change)}
\begin{figure}
  \centering
  \begin{columns}
    \column{0.4\textwidth}
    \centering
    \caption {Object of thought (abstract)}
    \includegraphics[width=0.99\textwidth]{number}
    \column{0.6\textwidth}
    \centering
    \caption {Object of some people's thought}
    \includegraphics[width=0.99\textwidth]{ufo}
  \end{columns}
\end{figure}
\end{frame}


\begin{frame}[fragile]{Hierarchy of being}
  \begin{enumerate}
  \item Change can happen only if the thing exists \vspace{5mm}
  \item But what keeps the thing in existence \emph{here and now}? \vspace{5mm}
  \item How something \emph{came to be} or how it might \emph{cease to exist} would not answer the question \vspace{5mm}
  \end{enumerate}
\end{frame}


\begin{frame}{Only `To Be' imparts being}
\begin{figure}
  \centering
  \begin{columns}
    %% \column{0.5\textwidth}
    %% \centering
    %% \caption {Can be many things...}
    %% \includegraphics[width=0.99\textwidth]{marble}
    \column{0.99\textwidth}
    \centering
    %% \caption {(but this is irrelevant)}
    \includegraphics[width=0.99\textwidth]{to_be}
  \end{columns}
\end{figure}
\end{frame}

\begin{frame}[fragile]{\emph{The} Unactualised Actualiser}
  \begin{enumerate}
  \item The first cause of existence must have no potential for existence \vspace{5mm}
  \item Pure actuality, then, could not have a cause of its own nor can it change \vspace{5mm}
  \item Time is a measure of change - therefore not in time \vspace{5mm}
  \item To be material entails being changeable - therefore not material \vspace{5mm}
  \end{enumerate}
\end{frame}


\section{Intelligence}


\begin{frame}{There's more to this}
\begin{figure}
  \centering
  \begin{columns}
    %% \column{0.5\textwidth}
    %% \centering
    %% \caption {Can be many things...}
    %% \includegraphics[width=0.99\textwidth]{marble}
    \column{0.99\textwidth}
    \centering
    %% \caption {(but this is irrelevant)}
    \includegraphics[width=0.99\textwidth]{abstract}
  \end{columns}
\end{figure}
\end{frame}


\begin{frame}[fragile]{To know abstract concepts}
  \begin{enumerate}
  \item Intelligence entails at least knowing the \emph{form} of things, i.e. what things are essentially \vspace{5mm}
  \item Whatever is in an effect, must in some way be in the totality of its cause \vspace{5mm}
  \item The cause of a thing's existence must also be the cause of the form of its existence \vspace{5mm}
  \end{enumerate}
\end{frame}


\section{Further reading}


\begin{frame}{Edward Feser}
\begin{figure}
  \centering
  \begin{columns}
    \column{0.32\textwidth}
    \centering
    %% \caption {In}
    \includegraphics[width=0.99\textwidth]{superstition}
    \column{0.32\textwidth}
    \centering
    %% \caption {this}
    \includegraphics[width=0.99\textwidth]{five_proofs}
    \column{0.34\textwidth}
    \centering
    %% \caption {order}
    \includegraphics[width=0.99\textwidth]{metaphysics}
  \end{columns}
\end{figure}
\end{frame}


\end{document}


%% \begin{frame}{}
%%   \centering
%%   \emph{If} something which causes a change is undergoing a change itself, \emph{then} that change requires a changer of its own
%% \end{frame}
