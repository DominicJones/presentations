% \documentclass{beamer}
\documentclass[xcolor=dvipsnames]{beamer}
%% \usecolortheme[named=Blue]{structure}
\setbeamersize{text margin left=30mm, text margin right=30mm}
\useoutertheme{infolines}
\usetheme[height=7mm]{Rochester}
\setbeamertemplate{items}[ball]
\setbeamertemplate{blocks}[rounded][shadow=true]
\setbeamertemplate{navigation symbols}{}

\usepackage[utf8x]{inputenc}
\usepackage{default}
\usepackage[english]{babel}
\usepackage{geometry}
%% \usepackage{fullpage}
\usepackage{amsmath, amsthm, amssymb}
\usepackage{listings}
\usepackage{pxfonts}
%% \usepackage{color}
%% \usepackage{graphicx}
%% \usepackage{natbib}
%% \usepackage{array}
%% \usepackage{booktabs}
%% \usepackage{tabu}
%% \usepackage[utf8]{inputenc}
%% \usepackage{fancyhdr}
%% \usepackage{float}
%% \usepackage{subfigure}
%% \usepackage{titlesec}


\definecolor{lstgray}{gray}{0.97}
\lstset{ %
  language=C++,
  basicstyle=\footnotesize\ttfamily,
  %% basicstyle=\ttfamily,
  %% keywordstyle=\color{blue}\ttfamily,
  keywordstyle=\bfseries,
  stringstyle=\color{red}\ttfamily,
  commentstyle=\color{green}\ttfamily,
  morecomment=[l][\color{magenta}]{\#},
  backgroundcolor=\color{lstgray},
  %% keywordstyle=\color{red},
  frame=f,
  frameround=ffff,
  tabsize=2,
  breaklines=true,
  breakatwhitespace=false,
  showspaces=false,
  showstringspaces=false,
  xleftmargin=5pt,
  xrightmargin=5pt,
  morekeywords={in,out,ref,auto,inout,import,ushort,scope,exit,mixin}
}

\def\sectionname{\translate{Section}}
\def\insertsectionnumber{\arabic{section}}
\setbeamertemplate{section page}
{
  \begin{centering}
    \begin{beamercolorbox}[sep=4pt,center]{part title}
      \usebeamerfont{section title}\insertsection\par
    \end{beamercolorbox}
  \end{centering}
}
\def\sectionpage{\usebeamertemplate*{section page}}


\AtBeginSection{\frame{\sectionpage}}


\title{Discrete Adjoint}
\subtitle{Another Perspective}
\author{Dominic Jones}
%% \institute{CD-adapco, London}
\date{December 2013}


\begin{document}
\begin{frame}[plain]
  \titlepage
\end{frame}


\section{Leveraging the Compiler}


\begin{frame}{General remarks}
\begin{itemize}
\item Compilers compile one translation unit at a time; to scale with compilation the generation of derivative code should follow the same design \vspace{5mm}
\item Compilers optimise intermediate code; generation of the derivative could benefit from this \vspace{5mm}
\item Functional programming lends itself to implementing derivatives of functions, to compiler optimisation and component programming
\end{itemize}
\end{frame}


\begin{frame}{Technology and trends}
\begin{itemize}
\item Composing a language from domain-specific embedded languages is an approach taken by Laurence Tratt at King’s College: instead of a language providing every feature, present a language which provides the tools to extend the language (but not beyond recognition) \vspace{5mm}
\item Template expressiveness and compile time function evaluation features are continually being developed \vspace{5mm}
\item The D programming language offers a handful of features which, when considered collectively, are very appealing for writing a compile-time embedded derivative code generator, or at the very least minimising the tediuum, code duplication and run-time costs of typical derivative implementations
\end{itemize}
\end{frame}


\begin{frame}{Roll your own -- Improved}
 D offers the following helping hands:
\vspace{5mm}
\begin{itemize}
\item \texttt{static if} \vspace{5mm}
\item \texttt{immutable} types \vspace{5mm}
\item \texttt{scope(exit)} \vspace{5mm}
\item compile time function evaluation via \texttt{static auto v = ...} \vspace{5mm}
\item \texttt{mixin template} \vspace{5mm}
\item \texttt{std.functional.memoize}
\end{itemize}
\end{frame}


\begin{frame}[fragile]
  \frametitle{Ingredients: scope (exit)}
Given some function, have the adjoint statements written in the same order as the primal statements. So if the function is originally:
\vspace{2mm}
  \begin{lstlisting}
void eval(in Primal a, in Primal b, out Primal c) {
  auto w1 = a * b;
  auto w2 = w1 * (a - b);
  c = w2 / w1;
}
  \end{lstlisting}
\vspace{2mm}
then after being differentiated, it may look something like...
\end{frame}


\begin{frame}[fragile]
  \frametitle{Ingredients: scope (exit)}
  \begin{lstlisting}
void eval(ref Dual a, ref Dual b, ref Dual c) {
  auto w1 = tuple!Dual(a[0] * b[0], 0);
  scope (exit) {
    b[1] += a[0] * w1[1];
    a[1] += b[0] * w1[1];
  }

  auto w2 = tuple!Dual(w1[0] * (a[0] - b[0]), 0);
  scope (exit) {
    w1[1] += (a[0] - b[0]) * w2[1];
    a[1] -= b[0] * w2[1];
    b[1] += a[0] * w2[1];
  }

  c = tuple!Dual(w2[0] / w1[0], 0);
  scope (exit) {
    w2[1] += c[1] / w1[0];
    w1[1] -= w2[0] * c[1] / w1[0]^^2;
  }
}
  \end{lstlisting}
\end{frame}


\begin{frame}[fragile]
  \frametitle{Ingredients: static if}
If the primal, tangent and adjoint statement can be written in the same order (using scope-exit), then perhaps a single function can embody the three versions without run-time overhead
\vspace{2mm}
  \begin{lstlisting}
void eval(Mode, Type)(ref Type a, ref Type b, ref Type c) {
  auto w1 = Type(a * b);
  static if (Mode.stringof == 'Tangent') {
    w1[1] = a[0] * b[1] + a[1] * b[0];
  }
  else if (Mode.stringof == 'Adjoint') {
    scope (exit) {
      b[1] += a[0] * w1[1];
      a[1] += b[0] * w1[1];
    }
  }

  ...
}
  \end{lstlisting}
\end{frame}


\begin{frame}[fragile]
  \frametitle{Ingredients: CTFE and mixin}
Using scope-exit and static-if still looks clumsy. One clean up could then be to parse the primal statement, generating its tangent or adjoint and mixing it in to the code
\vspace{2mm}
  \begin{lstlisting}
void eval(Mode, Type)(ref Type a, ref Type b, ref Type c) {
  mixin(generate!Mode('auto w1 = Type(a * b);'));

  ...
}
  \end{lstlisting}
\vspace{2mm}
where \texttt{generate(Mode)(statement)} would return (as a string) the adjoint statements and primal statement if the mode is adjoint, tangent statement and primal statement for the tangent mode and the primal statement for only the primal mode.
\end{frame}


\begin{frame}[fragile]
  \frametitle{Ingredients: CTFE and mixin}
\begin{itemize}
\item If \texttt{generate(Mode)(statement)} is used just in a statement-wise way, the implementation of the function would be quite simple, relying on compiler introspection to fill in the type details. \vspace{5mm}
\item A more sophisiticated \texttt{generate} parser may operate on a whole function or module. In this case the parser would need to be able to parse the language properly. \texttt{https://github.com/PhilippeSigaud/Pegged} offers such a parser. \vspace{5mm}
\item Having code written as strings is hardly ideal.
\end{itemize}
\end{frame}


\begin{frame}[fragile]
  \frametitle{Ingredients: mixin modules}
It's perhaps better to generate the derivative code in a separate file rather than clutter original implementations of functions.
  \begin{lstlisting}
module solver;

void solve(in double x, out double f)
{
  ...
}
  \end{lstlisting}
However, this approach may require significant work to be implemented.
  \begin{lstlisting}
module solver_derivative;

mixin(generate!Tangent(import("solver.d")));
mixin(generate!Adjoint(import("solver.d")));

unittest {
  solve_tng(x, x_tng, f, f_tng);
  solve_adj(x, x_adj, f, f_adj);
}
  \end{lstlisting}
\end{frame}

\end{document}
